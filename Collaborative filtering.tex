\documentclass[11pt]{article}

    \usepackage[breakable]{tcolorbox}
    \usepackage{parskip} % Stop auto-indenting (to mimic markdown behaviour)
    
    \usepackage{iftex}
    \ifPDFTeX
    	\usepackage[T1]{fontenc}
    	\usepackage{mathpazo}
    \else
    	\usepackage{fontspec}
    \fi

    % Basic figure setup, for now with no caption control since it's done
    % automatically by Pandoc (which extracts ![](path) syntax from Markdown).
    \usepackage{graphicx}
    % Maintain compatibility with old templates. Remove in nbconvert 6.0
    \let\Oldincludegraphics\includegraphics
    % Ensure that by default, figures have no caption (until we provide a
    % proper Figure object with a Caption API and a way to capture that
    % in the conversion process - todo).
    \usepackage{caption}
    \DeclareCaptionFormat{nocaption}{}
    \captionsetup{format=nocaption,aboveskip=0pt,belowskip=0pt}

    \usepackage[Export]{adjustbox} % Used to constrain images to a maximum size
    \adjustboxset{max size={0.9\linewidth}{0.9\paperheight}}
    \usepackage{float}
    \floatplacement{figure}{H} % forces figures to be placed at the correct location
    \usepackage{xcolor} % Allow colors to be defined
    \usepackage{enumerate} % Needed for markdown enumerations to work
    \usepackage{geometry} % Used to adjust the document margins
    \usepackage{amsmath} % Equations
    \usepackage{amssymb} % Equations
    \usepackage{textcomp} % defines textquotesingle
    % Hack from http://tex.stackexchange.com/a/47451/13684:
    \AtBeginDocument{%
        \def\PYZsq{\textquotesingle}% Upright quotes in Pygmentized code
    }
    \usepackage{upquote} % Upright quotes for verbatim code
    \usepackage{eurosym} % defines \euro
    \usepackage[mathletters]{ucs} % Extended unicode (utf-8) support
    \usepackage{fancyvrb} % verbatim replacement that allows latex
    \usepackage{grffile} % extends the file name processing of package graphics 
                         % to support a larger range
    \makeatletter % fix for grffile with XeLaTeX
    \def\Gread@@xetex#1{%
      \IfFileExists{"\Gin@base".bb}%
      {\Gread@eps{\Gin@base.bb}}%
      {\Gread@@xetex@aux#1}%
    }
    \makeatother

    % The hyperref package gives us a pdf with properly built
    % internal navigation ('pdf bookmarks' for the table of contents,
    % internal cross-reference links, web links for URLs, etc.)
    \usepackage{hyperref}
    % The default LaTeX title has an obnoxious amount of whitespace. By default,
    % titling removes some of it. It also provides customization options.
    \usepackage{titling}
    \usepackage{longtable} % longtable support required by pandoc >1.10
    \usepackage{booktabs}  % table support for pandoc > 1.12.2
    \usepackage[inline]{enumitem} % IRkernel/repr support (it uses the enumerate* environment)
    \usepackage[normalem]{ulem} % ulem is needed to support strikethroughs (\sout)
                                % normalem makes italics be italics, not underlines
    \usepackage{mathrsfs}
    

    
    % Colors for the hyperref package
    \definecolor{urlcolor}{rgb}{0,.145,.698}
    \definecolor{linkcolor}{rgb}{.71,0.21,0.01}
    \definecolor{citecolor}{rgb}{.12,.54,.11}

    % ANSI colors
    \definecolor{ansi-black}{HTML}{3E424D}
    \definecolor{ansi-black-intense}{HTML}{282C36}
    \definecolor{ansi-red}{HTML}{E75C58}
    \definecolor{ansi-red-intense}{HTML}{B22B31}
    \definecolor{ansi-green}{HTML}{00A250}
    \definecolor{ansi-green-intense}{HTML}{007427}
    \definecolor{ansi-yellow}{HTML}{DDB62B}
    \definecolor{ansi-yellow-intense}{HTML}{B27D12}
    \definecolor{ansi-blue}{HTML}{208FFB}
    \definecolor{ansi-blue-intense}{HTML}{0065CA}
    \definecolor{ansi-magenta}{HTML}{D160C4}
    \definecolor{ansi-magenta-intense}{HTML}{A03196}
    \definecolor{ansi-cyan}{HTML}{60C6C8}
    \definecolor{ansi-cyan-intense}{HTML}{258F8F}
    \definecolor{ansi-white}{HTML}{C5C1B4}
    \definecolor{ansi-white-intense}{HTML}{A1A6B2}
    \definecolor{ansi-default-inverse-fg}{HTML}{FFFFFF}
    \definecolor{ansi-default-inverse-bg}{HTML}{000000}

    % commands and environments needed by pandoc snippets
    % extracted from the output of `pandoc -s`
    \providecommand{\tightlist}{%
      \setlength{\itemsep}{0pt}\setlength{\parskip}{0pt}}
    \DefineVerbatimEnvironment{Highlighting}{Verbatim}{commandchars=\\\{\}}
    % Add ',fontsize=\small' for more characters per line
    \newenvironment{Shaded}{}{}
    \newcommand{\KeywordTok}[1]{\textcolor[rgb]{0.00,0.44,0.13}{\textbf{{#1}}}}
    \newcommand{\DataTypeTok}[1]{\textcolor[rgb]{0.56,0.13,0.00}{{#1}}}
    \newcommand{\DecValTok}[1]{\textcolor[rgb]{0.25,0.63,0.44}{{#1}}}
    \newcommand{\BaseNTok}[1]{\textcolor[rgb]{0.25,0.63,0.44}{{#1}}}
    \newcommand{\FloatTok}[1]{\textcolor[rgb]{0.25,0.63,0.44}{{#1}}}
    \newcommand{\CharTok}[1]{\textcolor[rgb]{0.25,0.44,0.63}{{#1}}}
    \newcommand{\StringTok}[1]{\textcolor[rgb]{0.25,0.44,0.63}{{#1}}}
    \newcommand{\CommentTok}[1]{\textcolor[rgb]{0.38,0.63,0.69}{\textit{{#1}}}}
    \newcommand{\OtherTok}[1]{\textcolor[rgb]{0.00,0.44,0.13}{{#1}}}
    \newcommand{\AlertTok}[1]{\textcolor[rgb]{1.00,0.00,0.00}{\textbf{{#1}}}}
    \newcommand{\FunctionTok}[1]{\textcolor[rgb]{0.02,0.16,0.49}{{#1}}}
    \newcommand{\RegionMarkerTok}[1]{{#1}}
    \newcommand{\ErrorTok}[1]{\textcolor[rgb]{1.00,0.00,0.00}{\textbf{{#1}}}}
    \newcommand{\NormalTok}[1]{{#1}}
    
    % Additional commands for more recent versions of Pandoc
    \newcommand{\ConstantTok}[1]{\textcolor[rgb]{0.53,0.00,0.00}{{#1}}}
    \newcommand{\SpecialCharTok}[1]{\textcolor[rgb]{0.25,0.44,0.63}{{#1}}}
    \newcommand{\VerbatimStringTok}[1]{\textcolor[rgb]{0.25,0.44,0.63}{{#1}}}
    \newcommand{\SpecialStringTok}[1]{\textcolor[rgb]{0.73,0.40,0.53}{{#1}}}
    \newcommand{\ImportTok}[1]{{#1}}
    \newcommand{\DocumentationTok}[1]{\textcolor[rgb]{0.73,0.13,0.13}{\textit{{#1}}}}
    \newcommand{\AnnotationTok}[1]{\textcolor[rgb]{0.38,0.63,0.69}{\textbf{\textit{{#1}}}}}
    \newcommand{\CommentVarTok}[1]{\textcolor[rgb]{0.38,0.63,0.69}{\textbf{\textit{{#1}}}}}
    \newcommand{\VariableTok}[1]{\textcolor[rgb]{0.10,0.09,0.49}{{#1}}}
    \newcommand{\ControlFlowTok}[1]{\textcolor[rgb]{0.00,0.44,0.13}{\textbf{{#1}}}}
    \newcommand{\OperatorTok}[1]{\textcolor[rgb]{0.40,0.40,0.40}{{#1}}}
    \newcommand{\BuiltInTok}[1]{{#1}}
    \newcommand{\ExtensionTok}[1]{{#1}}
    \newcommand{\PreprocessorTok}[1]{\textcolor[rgb]{0.74,0.48,0.00}{{#1}}}
    \newcommand{\AttributeTok}[1]{\textcolor[rgb]{0.49,0.56,0.16}{{#1}}}
    \newcommand{\InformationTok}[1]{\textcolor[rgb]{0.38,0.63,0.69}{\textbf{\textit{{#1}}}}}
    \newcommand{\WarningTok}[1]{\textcolor[rgb]{0.38,0.63,0.69}{\textbf{\textit{{#1}}}}}
    
    
    % Define a nice break command that doesn't care if a line doesn't already
    % exist.
    \def\br{\hspace*{\fill} \\* }
    % Math Jax compatibility definitions
    \def\gt{>}
    \def\lt{<}
    \let\Oldtex\TeX
    \let\Oldlatex\LaTeX
    \renewcommand{\TeX}{\textrm{\Oldtex}}
    \renewcommand{\LaTeX}{\textrm{\Oldlatex}}
    % Document parameters
    % Document title
    \title{Collaborative filtering}
    
    
    
    
    
% Pygments definitions
\makeatletter
\def\PY@reset{\let\PY@it=\relax \let\PY@bf=\relax%
    \let\PY@ul=\relax \let\PY@tc=\relax%
    \let\PY@bc=\relax \let\PY@ff=\relax}
\def\PY@tok#1{\csname PY@tok@#1\endcsname}
\def\PY@toks#1+{\ifx\relax#1\empty\else%
    \PY@tok{#1}\expandafter\PY@toks\fi}
\def\PY@do#1{\PY@bc{\PY@tc{\PY@ul{%
    \PY@it{\PY@bf{\PY@ff{#1}}}}}}}
\def\PY#1#2{\PY@reset\PY@toks#1+\relax+\PY@do{#2}}

\expandafter\def\csname PY@tok@w\endcsname{\def\PY@tc##1{\textcolor[rgb]{0.73,0.73,0.73}{##1}}}
\expandafter\def\csname PY@tok@c\endcsname{\let\PY@it=\textit\def\PY@tc##1{\textcolor[rgb]{0.25,0.50,0.50}{##1}}}
\expandafter\def\csname PY@tok@cp\endcsname{\def\PY@tc##1{\textcolor[rgb]{0.74,0.48,0.00}{##1}}}
\expandafter\def\csname PY@tok@k\endcsname{\let\PY@bf=\textbf\def\PY@tc##1{\textcolor[rgb]{0.00,0.50,0.00}{##1}}}
\expandafter\def\csname PY@tok@kp\endcsname{\def\PY@tc##1{\textcolor[rgb]{0.00,0.50,0.00}{##1}}}
\expandafter\def\csname PY@tok@kt\endcsname{\def\PY@tc##1{\textcolor[rgb]{0.69,0.00,0.25}{##1}}}
\expandafter\def\csname PY@tok@o\endcsname{\def\PY@tc##1{\textcolor[rgb]{0.40,0.40,0.40}{##1}}}
\expandafter\def\csname PY@tok@ow\endcsname{\let\PY@bf=\textbf\def\PY@tc##1{\textcolor[rgb]{0.67,0.13,1.00}{##1}}}
\expandafter\def\csname PY@tok@nb\endcsname{\def\PY@tc##1{\textcolor[rgb]{0.00,0.50,0.00}{##1}}}
\expandafter\def\csname PY@tok@nf\endcsname{\def\PY@tc##1{\textcolor[rgb]{0.00,0.00,1.00}{##1}}}
\expandafter\def\csname PY@tok@nc\endcsname{\let\PY@bf=\textbf\def\PY@tc##1{\textcolor[rgb]{0.00,0.00,1.00}{##1}}}
\expandafter\def\csname PY@tok@nn\endcsname{\let\PY@bf=\textbf\def\PY@tc##1{\textcolor[rgb]{0.00,0.00,1.00}{##1}}}
\expandafter\def\csname PY@tok@ne\endcsname{\let\PY@bf=\textbf\def\PY@tc##1{\textcolor[rgb]{0.82,0.25,0.23}{##1}}}
\expandafter\def\csname PY@tok@nv\endcsname{\def\PY@tc##1{\textcolor[rgb]{0.10,0.09,0.49}{##1}}}
\expandafter\def\csname PY@tok@no\endcsname{\def\PY@tc##1{\textcolor[rgb]{0.53,0.00,0.00}{##1}}}
\expandafter\def\csname PY@tok@nl\endcsname{\def\PY@tc##1{\textcolor[rgb]{0.63,0.63,0.00}{##1}}}
\expandafter\def\csname PY@tok@ni\endcsname{\let\PY@bf=\textbf\def\PY@tc##1{\textcolor[rgb]{0.60,0.60,0.60}{##1}}}
\expandafter\def\csname PY@tok@na\endcsname{\def\PY@tc##1{\textcolor[rgb]{0.49,0.56,0.16}{##1}}}
\expandafter\def\csname PY@tok@nt\endcsname{\let\PY@bf=\textbf\def\PY@tc##1{\textcolor[rgb]{0.00,0.50,0.00}{##1}}}
\expandafter\def\csname PY@tok@nd\endcsname{\def\PY@tc##1{\textcolor[rgb]{0.67,0.13,1.00}{##1}}}
\expandafter\def\csname PY@tok@s\endcsname{\def\PY@tc##1{\textcolor[rgb]{0.73,0.13,0.13}{##1}}}
\expandafter\def\csname PY@tok@sd\endcsname{\let\PY@it=\textit\def\PY@tc##1{\textcolor[rgb]{0.73,0.13,0.13}{##1}}}
\expandafter\def\csname PY@tok@si\endcsname{\let\PY@bf=\textbf\def\PY@tc##1{\textcolor[rgb]{0.73,0.40,0.53}{##1}}}
\expandafter\def\csname PY@tok@se\endcsname{\let\PY@bf=\textbf\def\PY@tc##1{\textcolor[rgb]{0.73,0.40,0.13}{##1}}}
\expandafter\def\csname PY@tok@sr\endcsname{\def\PY@tc##1{\textcolor[rgb]{0.73,0.40,0.53}{##1}}}
\expandafter\def\csname PY@tok@ss\endcsname{\def\PY@tc##1{\textcolor[rgb]{0.10,0.09,0.49}{##1}}}
\expandafter\def\csname PY@tok@sx\endcsname{\def\PY@tc##1{\textcolor[rgb]{0.00,0.50,0.00}{##1}}}
\expandafter\def\csname PY@tok@m\endcsname{\def\PY@tc##1{\textcolor[rgb]{0.40,0.40,0.40}{##1}}}
\expandafter\def\csname PY@tok@gh\endcsname{\let\PY@bf=\textbf\def\PY@tc##1{\textcolor[rgb]{0.00,0.00,0.50}{##1}}}
\expandafter\def\csname PY@tok@gu\endcsname{\let\PY@bf=\textbf\def\PY@tc##1{\textcolor[rgb]{0.50,0.00,0.50}{##1}}}
\expandafter\def\csname PY@tok@gd\endcsname{\def\PY@tc##1{\textcolor[rgb]{0.63,0.00,0.00}{##1}}}
\expandafter\def\csname PY@tok@gi\endcsname{\def\PY@tc##1{\textcolor[rgb]{0.00,0.63,0.00}{##1}}}
\expandafter\def\csname PY@tok@gr\endcsname{\def\PY@tc##1{\textcolor[rgb]{1.00,0.00,0.00}{##1}}}
\expandafter\def\csname PY@tok@ge\endcsname{\let\PY@it=\textit}
\expandafter\def\csname PY@tok@gs\endcsname{\let\PY@bf=\textbf}
\expandafter\def\csname PY@tok@gp\endcsname{\let\PY@bf=\textbf\def\PY@tc##1{\textcolor[rgb]{0.00,0.00,0.50}{##1}}}
\expandafter\def\csname PY@tok@go\endcsname{\def\PY@tc##1{\textcolor[rgb]{0.53,0.53,0.53}{##1}}}
\expandafter\def\csname PY@tok@gt\endcsname{\def\PY@tc##1{\textcolor[rgb]{0.00,0.27,0.87}{##1}}}
\expandafter\def\csname PY@tok@err\endcsname{\def\PY@bc##1{\setlength{\fboxsep}{0pt}\fcolorbox[rgb]{1.00,0.00,0.00}{1,1,1}{\strut ##1}}}
\expandafter\def\csname PY@tok@kc\endcsname{\let\PY@bf=\textbf\def\PY@tc##1{\textcolor[rgb]{0.00,0.50,0.00}{##1}}}
\expandafter\def\csname PY@tok@kd\endcsname{\let\PY@bf=\textbf\def\PY@tc##1{\textcolor[rgb]{0.00,0.50,0.00}{##1}}}
\expandafter\def\csname PY@tok@kn\endcsname{\let\PY@bf=\textbf\def\PY@tc##1{\textcolor[rgb]{0.00,0.50,0.00}{##1}}}
\expandafter\def\csname PY@tok@kr\endcsname{\let\PY@bf=\textbf\def\PY@tc##1{\textcolor[rgb]{0.00,0.50,0.00}{##1}}}
\expandafter\def\csname PY@tok@bp\endcsname{\def\PY@tc##1{\textcolor[rgb]{0.00,0.50,0.00}{##1}}}
\expandafter\def\csname PY@tok@fm\endcsname{\def\PY@tc##1{\textcolor[rgb]{0.00,0.00,1.00}{##1}}}
\expandafter\def\csname PY@tok@vc\endcsname{\def\PY@tc##1{\textcolor[rgb]{0.10,0.09,0.49}{##1}}}
\expandafter\def\csname PY@tok@vg\endcsname{\def\PY@tc##1{\textcolor[rgb]{0.10,0.09,0.49}{##1}}}
\expandafter\def\csname PY@tok@vi\endcsname{\def\PY@tc##1{\textcolor[rgb]{0.10,0.09,0.49}{##1}}}
\expandafter\def\csname PY@tok@vm\endcsname{\def\PY@tc##1{\textcolor[rgb]{0.10,0.09,0.49}{##1}}}
\expandafter\def\csname PY@tok@sa\endcsname{\def\PY@tc##1{\textcolor[rgb]{0.73,0.13,0.13}{##1}}}
\expandafter\def\csname PY@tok@sb\endcsname{\def\PY@tc##1{\textcolor[rgb]{0.73,0.13,0.13}{##1}}}
\expandafter\def\csname PY@tok@sc\endcsname{\def\PY@tc##1{\textcolor[rgb]{0.73,0.13,0.13}{##1}}}
\expandafter\def\csname PY@tok@dl\endcsname{\def\PY@tc##1{\textcolor[rgb]{0.73,0.13,0.13}{##1}}}
\expandafter\def\csname PY@tok@s2\endcsname{\def\PY@tc##1{\textcolor[rgb]{0.73,0.13,0.13}{##1}}}
\expandafter\def\csname PY@tok@sh\endcsname{\def\PY@tc##1{\textcolor[rgb]{0.73,0.13,0.13}{##1}}}
\expandafter\def\csname PY@tok@s1\endcsname{\def\PY@tc##1{\textcolor[rgb]{0.73,0.13,0.13}{##1}}}
\expandafter\def\csname PY@tok@mb\endcsname{\def\PY@tc##1{\textcolor[rgb]{0.40,0.40,0.40}{##1}}}
\expandafter\def\csname PY@tok@mf\endcsname{\def\PY@tc##1{\textcolor[rgb]{0.40,0.40,0.40}{##1}}}
\expandafter\def\csname PY@tok@mh\endcsname{\def\PY@tc##1{\textcolor[rgb]{0.40,0.40,0.40}{##1}}}
\expandafter\def\csname PY@tok@mi\endcsname{\def\PY@tc##1{\textcolor[rgb]{0.40,0.40,0.40}{##1}}}
\expandafter\def\csname PY@tok@il\endcsname{\def\PY@tc##1{\textcolor[rgb]{0.40,0.40,0.40}{##1}}}
\expandafter\def\csname PY@tok@mo\endcsname{\def\PY@tc##1{\textcolor[rgb]{0.40,0.40,0.40}{##1}}}
\expandafter\def\csname PY@tok@ch\endcsname{\let\PY@it=\textit\def\PY@tc##1{\textcolor[rgb]{0.25,0.50,0.50}{##1}}}
\expandafter\def\csname PY@tok@cm\endcsname{\let\PY@it=\textit\def\PY@tc##1{\textcolor[rgb]{0.25,0.50,0.50}{##1}}}
\expandafter\def\csname PY@tok@cpf\endcsname{\let\PY@it=\textit\def\PY@tc##1{\textcolor[rgb]{0.25,0.50,0.50}{##1}}}
\expandafter\def\csname PY@tok@c1\endcsname{\let\PY@it=\textit\def\PY@tc##1{\textcolor[rgb]{0.25,0.50,0.50}{##1}}}
\expandafter\def\csname PY@tok@cs\endcsname{\let\PY@it=\textit\def\PY@tc##1{\textcolor[rgb]{0.25,0.50,0.50}{##1}}}

\def\PYZbs{\char`\\}
\def\PYZus{\char`\_}
\def\PYZob{\char`\{}
\def\PYZcb{\char`\}}
\def\PYZca{\char`\^}
\def\PYZam{\char`\&}
\def\PYZlt{\char`\<}
\def\PYZgt{\char`\>}
\def\PYZsh{\char`\#}
\def\PYZpc{\char`\%}
\def\PYZdl{\char`\$}
\def\PYZhy{\char`\-}
\def\PYZsq{\char`\'}
\def\PYZdq{\char`\"}
\def\PYZti{\char`\~}
% for compatibility with earlier versions
\def\PYZat{@}
\def\PYZlb{[}
\def\PYZrb{]}
\makeatother


    % For linebreaks inside Verbatim environment from package fancyvrb. 
    \makeatletter
        \newbox\Wrappedcontinuationbox 
        \newbox\Wrappedvisiblespacebox 
        \newcommand*\Wrappedvisiblespace {\textcolor{red}{\textvisiblespace}} 
        \newcommand*\Wrappedcontinuationsymbol {\textcolor{red}{\llap{\tiny$\m@th\hookrightarrow$}}} 
        \newcommand*\Wrappedcontinuationindent {3ex } 
        \newcommand*\Wrappedafterbreak {\kern\Wrappedcontinuationindent\copy\Wrappedcontinuationbox} 
        % Take advantage of the already applied Pygments mark-up to insert 
        % potential linebreaks for TeX processing. 
        %        {, <, #, %, $, ' and ": go to next line. 
        %        _, }, ^, &, >, - and ~: stay at end of broken line. 
        % Use of \textquotesingle for straight quote. 
        \newcommand*\Wrappedbreaksatspecials {% 
            \def\PYGZus{\discretionary{\char`\_}{\Wrappedafterbreak}{\char`\_}}% 
            \def\PYGZob{\discretionary{}{\Wrappedafterbreak\char`\{}{\char`\{}}% 
            \def\PYGZcb{\discretionary{\char`\}}{\Wrappedafterbreak}{\char`\}}}% 
            \def\PYGZca{\discretionary{\char`\^}{\Wrappedafterbreak}{\char`\^}}% 
            \def\PYGZam{\discretionary{\char`\&}{\Wrappedafterbreak}{\char`\&}}% 
            \def\PYGZlt{\discretionary{}{\Wrappedafterbreak\char`\<}{\char`\<}}% 
            \def\PYGZgt{\discretionary{\char`\>}{\Wrappedafterbreak}{\char`\>}}% 
            \def\PYGZsh{\discretionary{}{\Wrappedafterbreak\char`\#}{\char`\#}}% 
            \def\PYGZpc{\discretionary{}{\Wrappedafterbreak\char`\%}{\char`\%}}% 
            \def\PYGZdl{\discretionary{}{\Wrappedafterbreak\char`\$}{\char`\$}}% 
            \def\PYGZhy{\discretionary{\char`\-}{\Wrappedafterbreak}{\char`\-}}% 
            \def\PYGZsq{\discretionary{}{\Wrappedafterbreak\textquotesingle}{\textquotesingle}}% 
            \def\PYGZdq{\discretionary{}{\Wrappedafterbreak\char`\"}{\char`\"}}% 
            \def\PYGZti{\discretionary{\char`\~}{\Wrappedafterbreak}{\char`\~}}% 
        } 
        % Some characters . , ; ? ! / are not pygmentized. 
        % This macro makes them "active" and they will insert potential linebreaks 
        \newcommand*\Wrappedbreaksatpunct {% 
            \lccode`\~`\.\lowercase{\def~}{\discretionary{\hbox{\char`\.}}{\Wrappedafterbreak}{\hbox{\char`\.}}}% 
            \lccode`\~`\,\lowercase{\def~}{\discretionary{\hbox{\char`\,}}{\Wrappedafterbreak}{\hbox{\char`\,}}}% 
            \lccode`\~`\;\lowercase{\def~}{\discretionary{\hbox{\char`\;}}{\Wrappedafterbreak}{\hbox{\char`\;}}}% 
            \lccode`\~`\:\lowercase{\def~}{\discretionary{\hbox{\char`\:}}{\Wrappedafterbreak}{\hbox{\char`\:}}}% 
            \lccode`\~`\?\lowercase{\def~}{\discretionary{\hbox{\char`\?}}{\Wrappedafterbreak}{\hbox{\char`\?}}}% 
            \lccode`\~`\!\lowercase{\def~}{\discretionary{\hbox{\char`\!}}{\Wrappedafterbreak}{\hbox{\char`\!}}}% 
            \lccode`\~`\/\lowercase{\def~}{\discretionary{\hbox{\char`\/}}{\Wrappedafterbreak}{\hbox{\char`\/}}}% 
            \catcode`\.\active
            \catcode`\,\active 
            \catcode`\;\active
            \catcode`\:\active
            \catcode`\?\active
            \catcode`\!\active
            \catcode`\/\active 
            \lccode`\~`\~ 	
        }
    \makeatother

    \let\OriginalVerbatim=\Verbatim
    \makeatletter
    \renewcommand{\Verbatim}[1][1]{%
        %\parskip\z@skip
        \sbox\Wrappedcontinuationbox {\Wrappedcontinuationsymbol}%
        \sbox\Wrappedvisiblespacebox {\FV@SetupFont\Wrappedvisiblespace}%
        \def\FancyVerbFormatLine ##1{\hsize\linewidth
            \vtop{\raggedright\hyphenpenalty\z@\exhyphenpenalty\z@
                \doublehyphendemerits\z@\finalhyphendemerits\z@
                \strut ##1\strut}%
        }%
        % If the linebreak is at a space, the latter will be displayed as visible
        % space at end of first line, and a continuation symbol starts next line.
        % Stretch/shrink are however usually zero for typewriter font.
        \def\FV@Space {%
            \nobreak\hskip\z@ plus\fontdimen3\font minus\fontdimen4\font
            \discretionary{\copy\Wrappedvisiblespacebox}{\Wrappedafterbreak}
            {\kern\fontdimen2\font}%
        }%
        
        % Allow breaks at special characters using \PYG... macros.
        \Wrappedbreaksatspecials
        % Breaks at punctuation characters . , ; ? ! and / need catcode=\active 	
        \OriginalVerbatim[#1,codes*=\Wrappedbreaksatpunct]%
    }
    \makeatother

    % Exact colors from NB
    \definecolor{incolor}{HTML}{303F9F}
    \definecolor{outcolor}{HTML}{D84315}
    \definecolor{cellborder}{HTML}{CFCFCF}
    \definecolor{cellbackground}{HTML}{F7F7F7}
    
    % prompt
    \makeatletter
    \newcommand{\boxspacing}{\kern\kvtcb@left@rule\kern\kvtcb@boxsep}
    \makeatother
    \newcommand{\prompt}[4]{
        \ttfamily\llap{{\color{#2}[#3]:\hspace{3pt}#4}}\vspace{-\baselineskip}
    }
    

    
    % Prevent overflowing lines due to hard-to-break entities
    \sloppy 
    % Setup hyperref package
    \hypersetup{
      breaklinks=true,  % so long urls are correctly broken across lines
      colorlinks=true,
      urlcolor=urlcolor,
      linkcolor=linkcolor,
      citecolor=citecolor,
      }
    % Slightly bigger margins than the latex defaults
    
    \geometry{verbose,tmargin=1in,bmargin=1in,lmargin=1in,rmargin=1in}
    
    

\begin{document}
    
    \maketitle
    
    

    
    COLLABORATIVE FILTERING

    Recommendation systems are a collection of algorithms used to recommend
items to users based on information taken from the user. These systems
have become ubiquitous can be commonly seen in online stores, movies
databases and job finders. In this notebook, we will explore
recommendation systems based on Collaborative Filtering and implement
simple version of one using Python and the Pandas library.

    Table of contents

\begin{verbatim}
<ol>
    <li><a href="#ref1">Acquiring the Data</a></li>
    <li><a href="#ref2">Preprocessing</a></li>
    <li><a href="#ref3">Collaborative Filtering</a></li>
</ol>
\end{verbatim}

    \# Acquiring the Data

    To acquire and extract the data, simply run the following Bash
scripts:\\
Dataset acquired from
\href{http://grouplens.org/datasets/movielens/}{GroupLens}. Lets
download the dataset. To download the data, we will use
\textbf{\texttt{!wget}} to download it from IBM Object Storage.\\
\textbf{Did you know?} When it comes to Machine Learning, you will
likely be working with large datasets. As a business, where can you host
your data? IBM is offering a unique opportunity for businesses, with 10
Tb of IBM Cloud Object Storage:
\href{http://cocl.us/ML0101EN-IBM-Offer-CC}{Sign up now for free}

    \begin{tcolorbox}[breakable, size=fbox, boxrule=1pt, pad at break*=1mm,colback=cellbackground, colframe=cellborder]
\prompt{In}{incolor}{1}{\boxspacing}
\begin{Verbatim}[commandchars=\\\{\}]
\PY{o}{!}wget \PYZhy{}O moviedataset.zip https://s3\PYZhy{}api.us\PYZhy{}geo.objectstorage.softlayer.net/cf\PYZhy{}courses\PYZhy{}data/CognitiveClass/ML0101ENv3/labs/moviedataset.zip
\PY{n+nb}{print}\PY{p}{(}\PY{l+s+s1}{\PYZsq{}}\PY{l+s+s1}{unziping ...}\PY{l+s+s1}{\PYZsq{}}\PY{p}{)}
\PY{o}{!}unzip \PYZhy{}o \PYZhy{}j moviedataset.zip 
\end{Verbatim}
\end{tcolorbox}

    \begin{Verbatim}[commandchars=\\\{\}]
--2020-11-01 04:02:31--  https://s3-api.us-geo.objectstorage.softlayer.net/cf-
courses-data/CognitiveClass/ML0101ENv3/labs/moviedataset.zip
Resolving s3-api.us-geo.objectstorage.softlayer.net (s3-api.us-
geo.objectstorage.softlayer.net){\ldots} 67.228.254.196
Connecting to s3-api.us-geo.objectstorage.softlayer.net (s3-api.us-
geo.objectstorage.softlayer.net)|67.228.254.196|:443{\ldots} connected.
HTTP request sent, awaiting response{\ldots} 200 OK
Length: 160301210 (153M) [application/zip]
Saving to: ‘moviedataset.zip’

moviedataset.zip    100\%[===================>] 152.88M  31.3MB/s    in 5.0s

2020-11-01 04:02:36 (30.5 MB/s) - ‘moviedataset.zip’ saved [160301210/160301210]

unziping {\ldots}
Archive:  moviedataset.zip
  inflating: links.csv
  inflating: movies.csv
  inflating: ratings.csv
  inflating: README.txt
  inflating: tags.csv
    \end{Verbatim}

    Now you're ready to start working with the data!

    \# Preprocessing

    First, let's get all of the imports out of the way:

    \begin{tcolorbox}[breakable, size=fbox, boxrule=1pt, pad at break*=1mm,colback=cellbackground, colframe=cellborder]
\prompt{In}{incolor}{2}{\boxspacing}
\begin{Verbatim}[commandchars=\\\{\}]
\PY{c+c1}{\PYZsh{}Dataframe manipulation library}
\PY{k+kn}{import} \PY{n+nn}{pandas} \PY{k}{as} \PY{n+nn}{pd}
\PY{c+c1}{\PYZsh{}Math functions, we\PYZsq{}ll only need the sqrt function so let\PYZsq{}s import only that}
\PY{k+kn}{from} \PY{n+nn}{math} \PY{k}{import} \PY{n}{sqrt}
\PY{k+kn}{import} \PY{n+nn}{numpy} \PY{k}{as} \PY{n+nn}{np}
\PY{k+kn}{import} \PY{n+nn}{matplotlib}\PY{n+nn}{.}\PY{n+nn}{pyplot} \PY{k}{as} \PY{n+nn}{plt}
\PY{o}{\PYZpc{}}\PY{k}{matplotlib} inline
\end{Verbatim}
\end{tcolorbox}

    Now let's read each file into their Dataframes:

    \begin{tcolorbox}[breakable, size=fbox, boxrule=1pt, pad at break*=1mm,colback=cellbackground, colframe=cellborder]
\prompt{In}{incolor}{3}{\boxspacing}
\begin{Verbatim}[commandchars=\\\{\}]
\PY{c+c1}{\PYZsh{}Storing the movie information into a pandas dataframe}
\PY{n}{movies\PYZus{}df} \PY{o}{=} \PY{n}{pd}\PY{o}{.}\PY{n}{read\PYZus{}csv}\PY{p}{(}\PY{l+s+s1}{\PYZsq{}}\PY{l+s+s1}{movies.csv}\PY{l+s+s1}{\PYZsq{}}\PY{p}{)}
\PY{c+c1}{\PYZsh{}Storing the user information into a pandas dataframe}
\PY{n}{ratings\PYZus{}df} \PY{o}{=} \PY{n}{pd}\PY{o}{.}\PY{n}{read\PYZus{}csv}\PY{p}{(}\PY{l+s+s1}{\PYZsq{}}\PY{l+s+s1}{ratings.csv}\PY{l+s+s1}{\PYZsq{}}\PY{p}{)}
\end{Verbatim}
\end{tcolorbox}

    Let's also take a peek at how each of them are organized:

    \begin{tcolorbox}[breakable, size=fbox, boxrule=1pt, pad at break*=1mm,colback=cellbackground, colframe=cellborder]
\prompt{In}{incolor}{4}{\boxspacing}
\begin{Verbatim}[commandchars=\\\{\}]
\PY{c+c1}{\PYZsh{}Head is a function that gets the first N rows of a dataframe. N\PYZsq{}s default is 5.}
\PY{n}{movies\PYZus{}df}\PY{o}{.}\PY{n}{head}\PY{p}{(}\PY{p}{)}
\end{Verbatim}
\end{tcolorbox}

            \begin{tcolorbox}[breakable, size=fbox, boxrule=.5pt, pad at break*=1mm, opacityfill=0]
\prompt{Out}{outcolor}{4}{\boxspacing}
\begin{Verbatim}[commandchars=\\\{\}]
   movieId                               title  \textbackslash{}
0        1                    Toy Story (1995)
1        2                      Jumanji (1995)
2        3             Grumpier Old Men (1995)
3        4            Waiting to Exhale (1995)
4        5  Father of the Bride Part II (1995)

                                        genres
0  Adventure|Animation|Children|Comedy|Fantasy
1                   Adventure|Children|Fantasy
2                               Comedy|Romance
3                         Comedy|Drama|Romance
4                                       Comedy
\end{Verbatim}
\end{tcolorbox}
        
    So each movie has a unique ID, a title with its release year along with
it (Which may contain unicode characters) and several different genres
in the same field. Let's remove the year from the title column and place
it into its own one by using the handy
\href{http://pandas.pydata.org/pandas-docs/stable/generated/pandas.Series.str.extract.html\#pandas.Series.str.extract}{extract}
function that Pandas has.

    Let's remove the year from the \textbf{title} column by using pandas'
replace function and store in a new \textbf{year} column.

    \begin{tcolorbox}[breakable, size=fbox, boxrule=1pt, pad at break*=1mm,colback=cellbackground, colframe=cellborder]
\prompt{In}{incolor}{5}{\boxspacing}
\begin{Verbatim}[commandchars=\\\{\}]
\PY{c+c1}{\PYZsh{}Using regular expressions to find a year stored between parentheses}
\PY{c+c1}{\PYZsh{}We specify the parantheses so we don\PYZsq{}t conflict with movies that have years in their titles}
\PY{n}{movies\PYZus{}df}\PY{p}{[}\PY{l+s+s1}{\PYZsq{}}\PY{l+s+s1}{year}\PY{l+s+s1}{\PYZsq{}}\PY{p}{]} \PY{o}{=} \PY{n}{movies\PYZus{}df}\PY{o}{.}\PY{n}{title}\PY{o}{.}\PY{n}{str}\PY{o}{.}\PY{n}{extract}\PY{p}{(}\PY{l+s+s1}{\PYZsq{}}\PY{l+s+s1}{(}\PY{l+s+s1}{\PYZbs{}}\PY{l+s+s1}{(}\PY{l+s+s1}{\PYZbs{}}\PY{l+s+s1}{d}\PY{l+s+s1}{\PYZbs{}}\PY{l+s+s1}{d}\PY{l+s+s1}{\PYZbs{}}\PY{l+s+s1}{d}\PY{l+s+s1}{\PYZbs{}}\PY{l+s+s1}{d}\PY{l+s+s1}{\PYZbs{}}\PY{l+s+s1}{))}\PY{l+s+s1}{\PYZsq{}}\PY{p}{,}\PY{n}{expand}\PY{o}{=}\PY{k+kc}{False}\PY{p}{)}
\PY{c+c1}{\PYZsh{}Removing the parentheses}
\PY{n}{movies\PYZus{}df}\PY{p}{[}\PY{l+s+s1}{\PYZsq{}}\PY{l+s+s1}{year}\PY{l+s+s1}{\PYZsq{}}\PY{p}{]} \PY{o}{=} \PY{n}{movies\PYZus{}df}\PY{o}{.}\PY{n}{year}\PY{o}{.}\PY{n}{str}\PY{o}{.}\PY{n}{extract}\PY{p}{(}\PY{l+s+s1}{\PYZsq{}}\PY{l+s+s1}{(}\PY{l+s+s1}{\PYZbs{}}\PY{l+s+s1}{d}\PY{l+s+s1}{\PYZbs{}}\PY{l+s+s1}{d}\PY{l+s+s1}{\PYZbs{}}\PY{l+s+s1}{d}\PY{l+s+s1}{\PYZbs{}}\PY{l+s+s1}{d)}\PY{l+s+s1}{\PYZsq{}}\PY{p}{,}\PY{n}{expand}\PY{o}{=}\PY{k+kc}{False}\PY{p}{)}
\PY{c+c1}{\PYZsh{}Removing the years from the \PYZsq{}title\PYZsq{} column}
\PY{n}{movies\PYZus{}df}\PY{p}{[}\PY{l+s+s1}{\PYZsq{}}\PY{l+s+s1}{title}\PY{l+s+s1}{\PYZsq{}}\PY{p}{]} \PY{o}{=} \PY{n}{movies\PYZus{}df}\PY{o}{.}\PY{n}{title}\PY{o}{.}\PY{n}{str}\PY{o}{.}\PY{n}{replace}\PY{p}{(}\PY{l+s+s1}{\PYZsq{}}\PY{l+s+s1}{(}\PY{l+s+s1}{\PYZbs{}}\PY{l+s+s1}{(}\PY{l+s+s1}{\PYZbs{}}\PY{l+s+s1}{d}\PY{l+s+s1}{\PYZbs{}}\PY{l+s+s1}{d}\PY{l+s+s1}{\PYZbs{}}\PY{l+s+s1}{d}\PY{l+s+s1}{\PYZbs{}}\PY{l+s+s1}{d}\PY{l+s+s1}{\PYZbs{}}\PY{l+s+s1}{))}\PY{l+s+s1}{\PYZsq{}}\PY{p}{,} \PY{l+s+s1}{\PYZsq{}}\PY{l+s+s1}{\PYZsq{}}\PY{p}{)}
\PY{c+c1}{\PYZsh{}Applying the strip function to get rid of any ending whitespace characters that may have appeared}
\PY{n}{movies\PYZus{}df}\PY{p}{[}\PY{l+s+s1}{\PYZsq{}}\PY{l+s+s1}{title}\PY{l+s+s1}{\PYZsq{}}\PY{p}{]} \PY{o}{=} \PY{n}{movies\PYZus{}df}\PY{p}{[}\PY{l+s+s1}{\PYZsq{}}\PY{l+s+s1}{title}\PY{l+s+s1}{\PYZsq{}}\PY{p}{]}\PY{o}{.}\PY{n}{apply}\PY{p}{(}\PY{k}{lambda} \PY{n}{x}\PY{p}{:} \PY{n}{x}\PY{o}{.}\PY{n}{strip}\PY{p}{(}\PY{p}{)}\PY{p}{)}
\end{Verbatim}
\end{tcolorbox}

    Let's look at the result!

    \begin{tcolorbox}[breakable, size=fbox, boxrule=1pt, pad at break*=1mm,colback=cellbackground, colframe=cellborder]
\prompt{In}{incolor}{6}{\boxspacing}
\begin{Verbatim}[commandchars=\\\{\}]
\PY{n}{movies\PYZus{}df}\PY{o}{.}\PY{n}{head}\PY{p}{(}\PY{p}{)}
\end{Verbatim}
\end{tcolorbox}

            \begin{tcolorbox}[breakable, size=fbox, boxrule=.5pt, pad at break*=1mm, opacityfill=0]
\prompt{Out}{outcolor}{6}{\boxspacing}
\begin{Verbatim}[commandchars=\\\{\}]
   movieId                        title  \textbackslash{}
0        1                    Toy Story
1        2                      Jumanji
2        3             Grumpier Old Men
3        4            Waiting to Exhale
4        5  Father of the Bride Part II

                                        genres  year
0  Adventure|Animation|Children|Comedy|Fantasy  1995
1                   Adventure|Children|Fantasy  1995
2                               Comedy|Romance  1995
3                         Comedy|Drama|Romance  1995
4                                       Comedy  1995
\end{Verbatim}
\end{tcolorbox}
        
    With that, let's also drop the genres column since we won't need it for
this particular recommendation system.

    \begin{tcolorbox}[breakable, size=fbox, boxrule=1pt, pad at break*=1mm,colback=cellbackground, colframe=cellborder]
\prompt{In}{incolor}{7}{\boxspacing}
\begin{Verbatim}[commandchars=\\\{\}]
\PY{c+c1}{\PYZsh{}Dropping the genres column}
\PY{n}{movies\PYZus{}df} \PY{o}{=} \PY{n}{movies\PYZus{}df}\PY{o}{.}\PY{n}{drop}\PY{p}{(}\PY{l+s+s1}{\PYZsq{}}\PY{l+s+s1}{genres}\PY{l+s+s1}{\PYZsq{}}\PY{p}{,} \PY{l+m+mi}{1}\PY{p}{)}
\end{Verbatim}
\end{tcolorbox}

    Here's the final movies dataframe:

    \begin{tcolorbox}[breakable, size=fbox, boxrule=1pt, pad at break*=1mm,colback=cellbackground, colframe=cellborder]
\prompt{In}{incolor}{8}{\boxspacing}
\begin{Verbatim}[commandchars=\\\{\}]
\PY{n}{movies\PYZus{}df}\PY{o}{.}\PY{n}{head}\PY{p}{(}\PY{p}{)}
\end{Verbatim}
\end{tcolorbox}

            \begin{tcolorbox}[breakable, size=fbox, boxrule=.5pt, pad at break*=1mm, opacityfill=0]
\prompt{Out}{outcolor}{8}{\boxspacing}
\begin{Verbatim}[commandchars=\\\{\}]
   movieId                        title  year
0        1                    Toy Story  1995
1        2                      Jumanji  1995
2        3             Grumpier Old Men  1995
3        4            Waiting to Exhale  1995
4        5  Father of the Bride Part II  1995
\end{Verbatim}
\end{tcolorbox}
        
    

    Next, let's look at the ratings dataframe.

    \begin{tcolorbox}[breakable, size=fbox, boxrule=1pt, pad at break*=1mm,colback=cellbackground, colframe=cellborder]
\prompt{In}{incolor}{9}{\boxspacing}
\begin{Verbatim}[commandchars=\\\{\}]
\PY{n}{ratings\PYZus{}df}\PY{o}{.}\PY{n}{head}\PY{p}{(}\PY{p}{)}
\end{Verbatim}
\end{tcolorbox}

            \begin{tcolorbox}[breakable, size=fbox, boxrule=.5pt, pad at break*=1mm, opacityfill=0]
\prompt{Out}{outcolor}{9}{\boxspacing}
\begin{Verbatim}[commandchars=\\\{\}]
   userId  movieId  rating   timestamp
0       1      169     2.5  1204927694
1       1     2471     3.0  1204927438
2       1    48516     5.0  1204927435
3       2     2571     3.5  1436165433
4       2   109487     4.0  1436165496
\end{Verbatim}
\end{tcolorbox}
        
    Every row in the ratings dataframe has a user id associated with at
least one movie, a rating and a timestamp showing when they reviewed it.
We won't be needing the timestamp column, so let's drop it to save on
memory.

    \begin{tcolorbox}[breakable, size=fbox, boxrule=1pt, pad at break*=1mm,colback=cellbackground, colframe=cellborder]
\prompt{In}{incolor}{10}{\boxspacing}
\begin{Verbatim}[commandchars=\\\{\}]
\PY{c+c1}{\PYZsh{}Drop removes a specified row or column from a dataframe}
\PY{n}{ratings\PYZus{}df} \PY{o}{=} \PY{n}{ratings\PYZus{}df}\PY{o}{.}\PY{n}{drop}\PY{p}{(}\PY{l+s+s1}{\PYZsq{}}\PY{l+s+s1}{timestamp}\PY{l+s+s1}{\PYZsq{}}\PY{p}{,} \PY{l+m+mi}{1}\PY{p}{)}
\end{Verbatim}
\end{tcolorbox}

    Here's how the final ratings Dataframe looks like:

    \begin{tcolorbox}[breakable, size=fbox, boxrule=1pt, pad at break*=1mm,colback=cellbackground, colframe=cellborder]
\prompt{In}{incolor}{11}{\boxspacing}
\begin{Verbatim}[commandchars=\\\{\}]
\PY{n}{ratings\PYZus{}df}\PY{o}{.}\PY{n}{head}\PY{p}{(}\PY{p}{)}
\end{Verbatim}
\end{tcolorbox}

            \begin{tcolorbox}[breakable, size=fbox, boxrule=.5pt, pad at break*=1mm, opacityfill=0]
\prompt{Out}{outcolor}{11}{\boxspacing}
\begin{Verbatim}[commandchars=\\\{\}]
   userId  movieId  rating
0       1      169     2.5
1       1     2471     3.0
2       1    48516     5.0
3       2     2571     3.5
4       2   109487     4.0
\end{Verbatim}
\end{tcolorbox}
        
    \# Collaborative Filtering

    Now, time to start our work on recommendation systems.

The first technique we're going to take a look at is called
\textbf{Collaborative Filtering}, which is also known as
\textbf{User-User Filtering}. As hinted by its alternate name, this
technique uses other users to recommend items to the input user. It
attempts to find users that have similar preferences and opinions as the
input and then recommends items that they have liked to the input. There
are several methods of finding similar users (Even some making use of
Machine Learning), and the one we will be using here is going to be
based on the \textbf{Pearson Correlation Function}.

\begin{figure}
\centering
\includegraphics{https://wikimedia.org/api/rest_v1/media/math/render/svg/bd1ccc2979b0fd1c1aec96e386f686ae874f9ec0}
\caption{alt text}
\end{figure}

The process for creating a User Based recommendation system is as
follows: - Select a user with the movies the user has watched - Based on
his rating to movies, find the top X neighbours - Get the watched movie
record of the user for each neighbour. - Calculate a similarity score
using some formula - Recommend the items with the highest score

Let's begin by creating an input user to recommend movies to:

Notice: To add more movies, simply increase the amount of elements in
the userInput. Feel free to add more in! Just be sure to write it in
with capital letters and if a movie starts with a ``The'', like ``The
Matrix'' then write it in like this: `Matrix, The' .

    \begin{tcolorbox}[breakable, size=fbox, boxrule=1pt, pad at break*=1mm,colback=cellbackground, colframe=cellborder]
\prompt{In}{incolor}{12}{\boxspacing}
\begin{Verbatim}[commandchars=\\\{\}]
\PY{n}{userInput} \PY{o}{=} \PY{p}{[}
            \PY{p}{\PYZob{}}\PY{l+s+s1}{\PYZsq{}}\PY{l+s+s1}{title}\PY{l+s+s1}{\PYZsq{}}\PY{p}{:}\PY{l+s+s1}{\PYZsq{}}\PY{l+s+s1}{Breakfast Club, The}\PY{l+s+s1}{\PYZsq{}}\PY{p}{,} \PY{l+s+s1}{\PYZsq{}}\PY{l+s+s1}{rating}\PY{l+s+s1}{\PYZsq{}}\PY{p}{:}\PY{l+m+mi}{5}\PY{p}{\PYZcb{}}\PY{p}{,}
            \PY{p}{\PYZob{}}\PY{l+s+s1}{\PYZsq{}}\PY{l+s+s1}{title}\PY{l+s+s1}{\PYZsq{}}\PY{p}{:}\PY{l+s+s1}{\PYZsq{}}\PY{l+s+s1}{Toy Story}\PY{l+s+s1}{\PYZsq{}}\PY{p}{,} \PY{l+s+s1}{\PYZsq{}}\PY{l+s+s1}{rating}\PY{l+s+s1}{\PYZsq{}}\PY{p}{:}\PY{l+m+mf}{3.5}\PY{p}{\PYZcb{}}\PY{p}{,}
            \PY{p}{\PYZob{}}\PY{l+s+s1}{\PYZsq{}}\PY{l+s+s1}{title}\PY{l+s+s1}{\PYZsq{}}\PY{p}{:}\PY{l+s+s1}{\PYZsq{}}\PY{l+s+s1}{Jumanji}\PY{l+s+s1}{\PYZsq{}}\PY{p}{,} \PY{l+s+s1}{\PYZsq{}}\PY{l+s+s1}{rating}\PY{l+s+s1}{\PYZsq{}}\PY{p}{:}\PY{l+m+mi}{2}\PY{p}{\PYZcb{}}\PY{p}{,}
            \PY{p}{\PYZob{}}\PY{l+s+s1}{\PYZsq{}}\PY{l+s+s1}{title}\PY{l+s+s1}{\PYZsq{}}\PY{p}{:}\PY{l+s+s2}{\PYZdq{}}\PY{l+s+s2}{Pulp Fiction}\PY{l+s+s2}{\PYZdq{}}\PY{p}{,} \PY{l+s+s1}{\PYZsq{}}\PY{l+s+s1}{rating}\PY{l+s+s1}{\PYZsq{}}\PY{p}{:}\PY{l+m+mi}{5}\PY{p}{\PYZcb{}}\PY{p}{,}
            \PY{p}{\PYZob{}}\PY{l+s+s1}{\PYZsq{}}\PY{l+s+s1}{title}\PY{l+s+s1}{\PYZsq{}}\PY{p}{:}\PY{l+s+s1}{\PYZsq{}}\PY{l+s+s1}{Akira}\PY{l+s+s1}{\PYZsq{}}\PY{p}{,} \PY{l+s+s1}{\PYZsq{}}\PY{l+s+s1}{rating}\PY{l+s+s1}{\PYZsq{}}\PY{p}{:}\PY{l+m+mf}{4.5}\PY{p}{\PYZcb{}}
         \PY{p}{]} 
\PY{n}{inputMovies} \PY{o}{=} \PY{n}{pd}\PY{o}{.}\PY{n}{DataFrame}\PY{p}{(}\PY{n}{userInput}\PY{p}{)}
\PY{n}{inputMovies}
\end{Verbatim}
\end{tcolorbox}

            \begin{tcolorbox}[breakable, size=fbox, boxrule=.5pt, pad at break*=1mm, opacityfill=0]
\prompt{Out}{outcolor}{12}{\boxspacing}
\begin{Verbatim}[commandchars=\\\{\}]
                 title  rating
0  Breakfast Club, The     5.0
1            Toy Story     3.5
2              Jumanji     2.0
3         Pulp Fiction     5.0
4                Akira     4.5
\end{Verbatim}
\end{tcolorbox}
        
    \hypertarget{add-movieid-to-input-user}{%
\paragraph{Add movieId to input user}\label{add-movieid-to-input-user}}

With the input complete, let's extract the input movies's ID's from the
movies dataframe and add them into it.

We can achieve this by first filtering out the rows that contain the
input movies' title and then merging this subset with the input
dataframe. We also drop unnecessary columns for the input to save memory
space.

    \begin{tcolorbox}[breakable, size=fbox, boxrule=1pt, pad at break*=1mm,colback=cellbackground, colframe=cellborder]
\prompt{In}{incolor}{13}{\boxspacing}
\begin{Verbatim}[commandchars=\\\{\}]
\PY{c+c1}{\PYZsh{}Filtering out the movies by title}
\PY{n}{inputId} \PY{o}{=} \PY{n}{movies\PYZus{}df}\PY{p}{[}\PY{n}{movies\PYZus{}df}\PY{p}{[}\PY{l+s+s1}{\PYZsq{}}\PY{l+s+s1}{title}\PY{l+s+s1}{\PYZsq{}}\PY{p}{]}\PY{o}{.}\PY{n}{isin}\PY{p}{(}\PY{n}{inputMovies}\PY{p}{[}\PY{l+s+s1}{\PYZsq{}}\PY{l+s+s1}{title}\PY{l+s+s1}{\PYZsq{}}\PY{p}{]}\PY{o}{.}\PY{n}{tolist}\PY{p}{(}\PY{p}{)}\PY{p}{)}\PY{p}{]}
\PY{c+c1}{\PYZsh{}Then merging it so we can get the movieId. It\PYZsq{}s implicitly merging it by title.}
\PY{n}{inputMovies} \PY{o}{=} \PY{n}{pd}\PY{o}{.}\PY{n}{merge}\PY{p}{(}\PY{n}{inputId}\PY{p}{,} \PY{n}{inputMovies}\PY{p}{)}
\PY{c+c1}{\PYZsh{}Dropping information we won\PYZsq{}t use from the input dataframe}
\PY{n}{inputMovies} \PY{o}{=} \PY{n}{inputMovies}\PY{o}{.}\PY{n}{drop}\PY{p}{(}\PY{l+s+s1}{\PYZsq{}}\PY{l+s+s1}{year}\PY{l+s+s1}{\PYZsq{}}\PY{p}{,} \PY{l+m+mi}{1}\PY{p}{)}
\PY{c+c1}{\PYZsh{}Final input dataframe}
\PY{c+c1}{\PYZsh{}If a movie you added in above isn\PYZsq{}t here, then it might not be in the original }
\PY{c+c1}{\PYZsh{}dataframe or it might spelled differently, please check capitalisation.}
\PY{n}{inputMovies}
\end{Verbatim}
\end{tcolorbox}

            \begin{tcolorbox}[breakable, size=fbox, boxrule=.5pt, pad at break*=1mm, opacityfill=0]
\prompt{Out}{outcolor}{13}{\boxspacing}
\begin{Verbatim}[commandchars=\\\{\}]
   movieId                title  rating
0        1            Toy Story     3.5
1        2              Jumanji     2.0
2      296         Pulp Fiction     5.0
3     1274                Akira     4.5
4     1968  Breakfast Club, The     5.0
\end{Verbatim}
\end{tcolorbox}
        
    \hypertarget{the-users-who-has-seen-the-same-movies}{%
\paragraph{The users who has seen the same
movies}\label{the-users-who-has-seen-the-same-movies}}

Now with the movie ID's in our input, we can now get the subset of users
that have watched and reviewed the movies in our input.

    \begin{tcolorbox}[breakable, size=fbox, boxrule=1pt, pad at break*=1mm,colback=cellbackground, colframe=cellborder]
\prompt{In}{incolor}{14}{\boxspacing}
\begin{Verbatim}[commandchars=\\\{\}]
\PY{c+c1}{\PYZsh{}Filtering out users that have watched movies that the input has watched and storing it}
\PY{n}{userSubset} \PY{o}{=} \PY{n}{ratings\PYZus{}df}\PY{p}{[}\PY{n}{ratings\PYZus{}df}\PY{p}{[}\PY{l+s+s1}{\PYZsq{}}\PY{l+s+s1}{movieId}\PY{l+s+s1}{\PYZsq{}}\PY{p}{]}\PY{o}{.}\PY{n}{isin}\PY{p}{(}\PY{n}{inputMovies}\PY{p}{[}\PY{l+s+s1}{\PYZsq{}}\PY{l+s+s1}{movieId}\PY{l+s+s1}{\PYZsq{}}\PY{p}{]}\PY{o}{.}\PY{n}{tolist}\PY{p}{(}\PY{p}{)}\PY{p}{)}\PY{p}{]}
\PY{n}{userSubset}\PY{o}{.}\PY{n}{head}\PY{p}{(}\PY{p}{)}
\end{Verbatim}
\end{tcolorbox}

            \begin{tcolorbox}[breakable, size=fbox, boxrule=.5pt, pad at break*=1mm, opacityfill=0]
\prompt{Out}{outcolor}{14}{\boxspacing}
\begin{Verbatim}[commandchars=\\\{\}]
     userId  movieId  rating
19        4      296     4.0
441      12     1968     3.0
479      13        2     2.0
531      13     1274     5.0
681      14      296     2.0
\end{Verbatim}
\end{tcolorbox}
        
    We now group up the rows by user ID.

    \begin{tcolorbox}[breakable, size=fbox, boxrule=1pt, pad at break*=1mm,colback=cellbackground, colframe=cellborder]
\prompt{In}{incolor}{15}{\boxspacing}
\begin{Verbatim}[commandchars=\\\{\}]
\PY{c+c1}{\PYZsh{}Groupby creates several sub dataframes where they all have the same value in the column specified as the parameter}
\PY{n}{userSubsetGroup} \PY{o}{=} \PY{n}{userSubset}\PY{o}{.}\PY{n}{groupby}\PY{p}{(}\PY{p}{[}\PY{l+s+s1}{\PYZsq{}}\PY{l+s+s1}{userId}\PY{l+s+s1}{\PYZsq{}}\PY{p}{]}\PY{p}{)}
\end{Verbatim}
\end{tcolorbox}

    lets look at one of the users, e.g.~the one with userID=1130

    \begin{tcolorbox}[breakable, size=fbox, boxrule=1pt, pad at break*=1mm,colback=cellbackground, colframe=cellborder]
\prompt{In}{incolor}{16}{\boxspacing}
\begin{Verbatim}[commandchars=\\\{\}]
\PY{n}{userSubsetGroup}\PY{o}{.}\PY{n}{get\PYZus{}group}\PY{p}{(}\PY{l+m+mi}{1130}\PY{p}{)}
\end{Verbatim}
\end{tcolorbox}

            \begin{tcolorbox}[breakable, size=fbox, boxrule=.5pt, pad at break*=1mm, opacityfill=0]
\prompt{Out}{outcolor}{16}{\boxspacing}
\begin{Verbatim}[commandchars=\\\{\}]
        userId  movieId  rating
104167    1130        1     0.5
104168    1130        2     4.0
104214    1130      296     4.0
104363    1130     1274     4.5
104443    1130     1968     4.5
\end{Verbatim}
\end{tcolorbox}
        
    Let's also sort these groups so the users that share the most movies in
common with the input have higher priority. This provides a richer
recommendation since we won't go through every single user.

    \begin{tcolorbox}[breakable, size=fbox, boxrule=1pt, pad at break*=1mm,colback=cellbackground, colframe=cellborder]
\prompt{In}{incolor}{17}{\boxspacing}
\begin{Verbatim}[commandchars=\\\{\}]
\PY{c+c1}{\PYZsh{}Sorting it so users with movie most in common with the input will have priority}
\PY{n}{userSubsetGroup} \PY{o}{=} \PY{n+nb}{sorted}\PY{p}{(}\PY{n}{userSubsetGroup}\PY{p}{,}  \PY{n}{key}\PY{o}{=}\PY{k}{lambda} \PY{n}{x}\PY{p}{:} \PY{n+nb}{len}\PY{p}{(}\PY{n}{x}\PY{p}{[}\PY{l+m+mi}{1}\PY{p}{]}\PY{p}{)}\PY{p}{,} \PY{n}{reverse}\PY{o}{=}\PY{k+kc}{True}\PY{p}{)}
\end{Verbatim}
\end{tcolorbox}

    Now lets look at the first user

    \begin{tcolorbox}[breakable, size=fbox, boxrule=1pt, pad at break*=1mm,colback=cellbackground, colframe=cellborder]
\prompt{In}{incolor}{18}{\boxspacing}
\begin{Verbatim}[commandchars=\\\{\}]
\PY{n}{userSubsetGroup}\PY{p}{[}\PY{l+m+mi}{0}\PY{p}{:}\PY{l+m+mi}{3}\PY{p}{]}
\end{Verbatim}
\end{tcolorbox}

            \begin{tcolorbox}[breakable, size=fbox, boxrule=.5pt, pad at break*=1mm, opacityfill=0]
\prompt{Out}{outcolor}{18}{\boxspacing}
\begin{Verbatim}[commandchars=\\\{\}]
[(75,
        userId  movieId  rating
  7507      75        1     5.0
  7508      75        2     3.5
  7540      75      296     5.0
  7633      75     1274     4.5
  7673      75     1968     5.0),
 (106,
        userId  movieId  rating
  9083     106        1     2.5
  9084     106        2     3.0
  9115     106      296     3.5
  9198     106     1274     3.0
  9238     106     1968     3.5),
 (686,
         userId  movieId  rating
  61336     686        1     4.0
  61337     686        2     3.0
  61377     686      296     4.0
  61478     686     1274     4.0
  61569     686     1968     5.0)]
\end{Verbatim}
\end{tcolorbox}
        
    \hypertarget{similarity-of-users-to-input-user}{%
\paragraph{Similarity of users to input
user}\label{similarity-of-users-to-input-user}}

Next, we are going to compare all users (not really all !!!) to our
specified user and find the one that is most similar.\\
we're going to find out how similar each user is to the input through
the \textbf{Pearson Correlation Coefficient}. It is used to measure the
strength of a linear association between two variables. The formula for
finding this coefficient between sets X and Y with N values can be seen
in the image below.

Why Pearson Correlation?

Pearson correlation is invariant to scaling, i.e.~multiplying all
elements by a nonzero constant or adding any constant to all elements.
For example, if you have two vectors X and Y,then, pearson(X, Y) ==
pearson(X, 2 * Y + 3). This is a pretty important property in
recommendation systems because for example two users might rate two
series of items totally different in terms of absolute rates, but they
would be similar users (i.e.~with similar ideas) with similar rates in
various scales .

\begin{figure}
\centering
\includegraphics{https://wikimedia.org/api/rest_v1/media/math/render/svg/bd1ccc2979b0fd1c1aec96e386f686ae874f9ec0}
\caption{alt text}
\end{figure}

The values given by the formula vary from r = -1 to r = 1, where 1 forms
a direct correlation between the two entities (it means a perfect
positive correlation) and -1 forms a perfect negative correlation.

In our case, a 1 means that the two users have similar tastes while a -1
means the opposite.

    We will select a subset of users to iterate through. This limit is
imposed because we don't want to waste too much time going through every
single user.

    \begin{tcolorbox}[breakable, size=fbox, boxrule=1pt, pad at break*=1mm,colback=cellbackground, colframe=cellborder]
\prompt{In}{incolor}{19}{\boxspacing}
\begin{Verbatim}[commandchars=\\\{\}]
\PY{n}{userSubsetGroup} \PY{o}{=} \PY{n}{userSubsetGroup}\PY{p}{[}\PY{l+m+mi}{0}\PY{p}{:}\PY{l+m+mi}{100}\PY{p}{]}
\end{Verbatim}
\end{tcolorbox}

    Now, we calculate the Pearson Correlation between input user and subset
group, and store it in a dictionary, where the key is the user Id and
the value is the coefficient

    \begin{tcolorbox}[breakable, size=fbox, boxrule=1pt, pad at break*=1mm,colback=cellbackground, colframe=cellborder]
\prompt{In}{incolor}{20}{\boxspacing}
\begin{Verbatim}[commandchars=\\\{\}]
\PY{c+c1}{\PYZsh{}Store the Pearson Correlation in a dictionary, where the key is the user Id and the value is the coefficient}
\PY{n}{pearsonCorrelationDict} \PY{o}{=} \PY{p}{\PYZob{}}\PY{p}{\PYZcb{}}

\PY{c+c1}{\PYZsh{}For every user group in our subset}
\PY{k}{for} \PY{n}{name}\PY{p}{,} \PY{n}{group} \PY{o+ow}{in} \PY{n}{userSubsetGroup}\PY{p}{:}
    \PY{c+c1}{\PYZsh{}Let\PYZsq{}s start by sorting the input and current user group so the values aren\PYZsq{}t mixed up later on}
    \PY{n}{group} \PY{o}{=} \PY{n}{group}\PY{o}{.}\PY{n}{sort\PYZus{}values}\PY{p}{(}\PY{n}{by}\PY{o}{=}\PY{l+s+s1}{\PYZsq{}}\PY{l+s+s1}{movieId}\PY{l+s+s1}{\PYZsq{}}\PY{p}{)}
    \PY{n}{inputMovies} \PY{o}{=} \PY{n}{inputMovies}\PY{o}{.}\PY{n}{sort\PYZus{}values}\PY{p}{(}\PY{n}{by}\PY{o}{=}\PY{l+s+s1}{\PYZsq{}}\PY{l+s+s1}{movieId}\PY{l+s+s1}{\PYZsq{}}\PY{p}{)}
    \PY{c+c1}{\PYZsh{}Get the N for the formula}
    \PY{n}{nRatings} \PY{o}{=} \PY{n+nb}{len}\PY{p}{(}\PY{n}{group}\PY{p}{)}
    \PY{c+c1}{\PYZsh{}Get the review scores for the movies that they both have in common}
    \PY{n}{temp\PYZus{}df} \PY{o}{=} \PY{n}{inputMovies}\PY{p}{[}\PY{n}{inputMovies}\PY{p}{[}\PY{l+s+s1}{\PYZsq{}}\PY{l+s+s1}{movieId}\PY{l+s+s1}{\PYZsq{}}\PY{p}{]}\PY{o}{.}\PY{n}{isin}\PY{p}{(}\PY{n}{group}\PY{p}{[}\PY{l+s+s1}{\PYZsq{}}\PY{l+s+s1}{movieId}\PY{l+s+s1}{\PYZsq{}}\PY{p}{]}\PY{o}{.}\PY{n}{tolist}\PY{p}{(}\PY{p}{)}\PY{p}{)}\PY{p}{]}
    \PY{c+c1}{\PYZsh{}And then store them in a temporary buffer variable in a list format to facilitate future calculations}
    \PY{n}{tempRatingList} \PY{o}{=} \PY{n}{temp\PYZus{}df}\PY{p}{[}\PY{l+s+s1}{\PYZsq{}}\PY{l+s+s1}{rating}\PY{l+s+s1}{\PYZsq{}}\PY{p}{]}\PY{o}{.}\PY{n}{tolist}\PY{p}{(}\PY{p}{)}
    \PY{c+c1}{\PYZsh{}Let\PYZsq{}s also put the current user group reviews in a list format}
    \PY{n}{tempGroupList} \PY{o}{=} \PY{n}{group}\PY{p}{[}\PY{l+s+s1}{\PYZsq{}}\PY{l+s+s1}{rating}\PY{l+s+s1}{\PYZsq{}}\PY{p}{]}\PY{o}{.}\PY{n}{tolist}\PY{p}{(}\PY{p}{)}
    \PY{c+c1}{\PYZsh{}Now let\PYZsq{}s calculate the pearson correlation between two users, so called, x and y}
    \PY{n}{Sxx} \PY{o}{=} \PY{n+nb}{sum}\PY{p}{(}\PY{p}{[}\PY{n}{i}\PY{o}{*}\PY{o}{*}\PY{l+m+mi}{2} \PY{k}{for} \PY{n}{i} \PY{o+ow}{in} \PY{n}{tempRatingList}\PY{p}{]}\PY{p}{)} \PY{o}{\PYZhy{}} \PY{n+nb}{pow}\PY{p}{(}\PY{n+nb}{sum}\PY{p}{(}\PY{n}{tempRatingList}\PY{p}{)}\PY{p}{,}\PY{l+m+mi}{2}\PY{p}{)}\PY{o}{/}\PY{n+nb}{float}\PY{p}{(}\PY{n}{nRatings}\PY{p}{)}
    \PY{n}{Syy} \PY{o}{=} \PY{n+nb}{sum}\PY{p}{(}\PY{p}{[}\PY{n}{i}\PY{o}{*}\PY{o}{*}\PY{l+m+mi}{2} \PY{k}{for} \PY{n}{i} \PY{o+ow}{in} \PY{n}{tempGroupList}\PY{p}{]}\PY{p}{)} \PY{o}{\PYZhy{}} \PY{n+nb}{pow}\PY{p}{(}\PY{n+nb}{sum}\PY{p}{(}\PY{n}{tempGroupList}\PY{p}{)}\PY{p}{,}\PY{l+m+mi}{2}\PY{p}{)}\PY{o}{/}\PY{n+nb}{float}\PY{p}{(}\PY{n}{nRatings}\PY{p}{)}
    \PY{n}{Sxy} \PY{o}{=} \PY{n+nb}{sum}\PY{p}{(} \PY{n}{i}\PY{o}{*}\PY{n}{j} \PY{k}{for} \PY{n}{i}\PY{p}{,} \PY{n}{j} \PY{o+ow}{in} \PY{n+nb}{zip}\PY{p}{(}\PY{n}{tempRatingList}\PY{p}{,} \PY{n}{tempGroupList}\PY{p}{)}\PY{p}{)} \PY{o}{\PYZhy{}} \PY{n+nb}{sum}\PY{p}{(}\PY{n}{tempRatingList}\PY{p}{)}\PY{o}{*}\PY{n+nb}{sum}\PY{p}{(}\PY{n}{tempGroupList}\PY{p}{)}\PY{o}{/}\PY{n+nb}{float}\PY{p}{(}\PY{n}{nRatings}\PY{p}{)}
    
    \PY{c+c1}{\PYZsh{}If the denominator is different than zero, then divide, else, 0 correlation.}
    \PY{k}{if} \PY{n}{Sxx} \PY{o}{!=} \PY{l+m+mi}{0} \PY{o+ow}{and} \PY{n}{Syy} \PY{o}{!=} \PY{l+m+mi}{0}\PY{p}{:}
        \PY{n}{pearsonCorrelationDict}\PY{p}{[}\PY{n}{name}\PY{p}{]} \PY{o}{=} \PY{n}{Sxy}\PY{o}{/}\PY{n}{sqrt}\PY{p}{(}\PY{n}{Sxx}\PY{o}{*}\PY{n}{Syy}\PY{p}{)}
    \PY{k}{else}\PY{p}{:}
        \PY{n}{pearsonCorrelationDict}\PY{p}{[}\PY{n}{name}\PY{p}{]} \PY{o}{=} \PY{l+m+mi}{0}
\end{Verbatim}
\end{tcolorbox}

    \begin{tcolorbox}[breakable, size=fbox, boxrule=1pt, pad at break*=1mm,colback=cellbackground, colframe=cellborder]
\prompt{In}{incolor}{21}{\boxspacing}
\begin{Verbatim}[commandchars=\\\{\}]
\PY{n}{pearsonCorrelationDict}\PY{o}{.}\PY{n}{items}\PY{p}{(}\PY{p}{)}
\end{Verbatim}
\end{tcolorbox}

            \begin{tcolorbox}[breakable, size=fbox, boxrule=.5pt, pad at break*=1mm, opacityfill=0]
\prompt{Out}{outcolor}{21}{\boxspacing}
\begin{Verbatim}[commandchars=\\\{\}]
dict\_items([(75, 0.8272781516947562), (106, 0.5860090386731182), (686,
0.8320502943378437), (815, 0.5765566601970551), (1040, 0.9434563530497265),
(1130, 0.2891574659831201), (1502, 0.8770580193070299), (1599,
0.4385290096535153), (1625, 0.716114874039432), (1950, 0.179028718509858),
(2065, 0.4385290096535153), (2128, 0.5860090386731196), (2432,
0.1386750490563073), (2791, 0.8770580193070299), (2839, 0.8204126541423674),
(2948, -0.11720180773462392), (3025, 0.45124262819713973), (3040,
0.89514359254929), (3186, 0.6784622064861935), (3271, 0.26989594817970664),
(3429, 0.0), (3734, -0.15041420939904673), (4099, 0.05860090386731196), (4208,
0.29417420270727607), (4282, -0.4385290096535115), (4292, 0.6564386345361464),
(4415, -0.11183835382312353), (4586, -0.9024852563942795), (4725,
-0.08006407690254357), (4818, 0.4885967564883424), (5104, 0.7674257668936507),
(5165, -0.4385290096535153), (5547, 0.17200522903844556), (6082,
-0.04728779924109591), (6207, 0.9615384615384616), (6366, 0.6577935144802716),
(6482, 0.0), (6530, -0.3516054232038709), (7235, 0.6981407669689391), (7403,
0.11720180773462363), (7641, 0.7161148740394331), (7996, 0.626600514784504),
(8008, -0.22562131409856986), (8086, 0.6933752452815365), (8245, 0.0), (8572,
0.8600261451922278), (8675, 0.5370861555295773), (9101, -0.08600261451922278),
(9358, 0.692178738358485), (9663, 0.193972725041952), (9994,
0.5030272728659587), (10248, -0.24806946917841693), (10315, 0.537086155529574),
(10368, 0.4688072309384945), (10607, 0.41602514716892186), (10707,
0.9615384615384616), (10863, 0.6020183016345595), (11314, 0.8204126541423654),
(11399, 0.517260600111872), (11769, 0.9376144618769914), (11827,
0.4902903378454601), (12069, 0.0), (12120, 0.9292940047327363), (12211,
0.8600261451922278), (12325, 0.9616783115081544), (12916, 0.5860090386731196),
(12921, 0.6611073566849309), (13053, 0.9607689228305227), (13142,
0.6016568375961863), (13260, 0.7844645405527362), (13366, 0.8951435925492911),
(13768, 0.8770580193070289), (13888, 0.2508726030021272), (13923,
0.3516054232038718), (13934, 0.17200522903844556), (14529, 0.7417901772340937),
(14551, 0.537086155529574), (14588, 0.21926450482675766), (14984,
0.716114874039432), (15137, 0.5860090386731196), (15157, 0.9035841064985974),
(15466, 0.7205766921228921), (15670, 0.516015687115336), (15834,
0.22562131409856986), (16292, 0.6577935144802716), (16456, 0.7161148740394331),
(16506, 0.5481612620668942), (17246, 0.48038446141526137), (17438,
0.7093169886164387), (17501, 0.8168748513121271), (17502, 0.8272781516947562),
(17666, 0.7689238340176859), (17735, 0.7042381820123422), (17742,
0.3922322702763681), (17757, 0.64657575013984), (17854, 0.537086155529574),
(17897, 0.8770580193070289), (17944, 0.2713848825944774), (18301,
0.29838119751643016), (18509, 0.1322214713369862)])
\end{Verbatim}
\end{tcolorbox}
        
    \begin{tcolorbox}[breakable, size=fbox, boxrule=1pt, pad at break*=1mm,colback=cellbackground, colframe=cellborder]
\prompt{In}{incolor}{22}{\boxspacing}
\begin{Verbatim}[commandchars=\\\{\}]
\PY{n}{pearsonDF} \PY{o}{=} \PY{n}{pd}\PY{o}{.}\PY{n}{DataFrame}\PY{o}{.}\PY{n}{from\PYZus{}dict}\PY{p}{(}\PY{n}{pearsonCorrelationDict}\PY{p}{,} \PY{n}{orient}\PY{o}{=}\PY{l+s+s1}{\PYZsq{}}\PY{l+s+s1}{index}\PY{l+s+s1}{\PYZsq{}}\PY{p}{)}
\PY{n}{pearsonDF}\PY{o}{.}\PY{n}{columns} \PY{o}{=} \PY{p}{[}\PY{l+s+s1}{\PYZsq{}}\PY{l+s+s1}{similarityIndex}\PY{l+s+s1}{\PYZsq{}}\PY{p}{]}
\PY{n}{pearsonDF}\PY{p}{[}\PY{l+s+s1}{\PYZsq{}}\PY{l+s+s1}{userId}\PY{l+s+s1}{\PYZsq{}}\PY{p}{]} \PY{o}{=} \PY{n}{pearsonDF}\PY{o}{.}\PY{n}{index}
\PY{n}{pearsonDF}\PY{o}{.}\PY{n}{index} \PY{o}{=} \PY{n+nb}{range}\PY{p}{(}\PY{n+nb}{len}\PY{p}{(}\PY{n}{pearsonDF}\PY{p}{)}\PY{p}{)}
\PY{n}{pearsonDF}\PY{o}{.}\PY{n}{head}\PY{p}{(}\PY{p}{)}
\end{Verbatim}
\end{tcolorbox}

            \begin{tcolorbox}[breakable, size=fbox, boxrule=.5pt, pad at break*=1mm, opacityfill=0]
\prompt{Out}{outcolor}{22}{\boxspacing}
\begin{Verbatim}[commandchars=\\\{\}]
   similarityIndex  userId
0         0.827278      75
1         0.586009     106
2         0.832050     686
3         0.576557     815
4         0.943456    1040
\end{Verbatim}
\end{tcolorbox}
        
    \hypertarget{the-top-x-similar-users-to-input-user}{%
\paragraph{The top x similar users to input
user}\label{the-top-x-similar-users-to-input-user}}

Now let's get the top 50 users that are most similar to the input.

    \begin{tcolorbox}[breakable, size=fbox, boxrule=1pt, pad at break*=1mm,colback=cellbackground, colframe=cellborder]
\prompt{In}{incolor}{23}{\boxspacing}
\begin{Verbatim}[commandchars=\\\{\}]
\PY{n}{topUsers}\PY{o}{=}\PY{n}{pearsonDF}\PY{o}{.}\PY{n}{sort\PYZus{}values}\PY{p}{(}\PY{n}{by}\PY{o}{=}\PY{l+s+s1}{\PYZsq{}}\PY{l+s+s1}{similarityIndex}\PY{l+s+s1}{\PYZsq{}}\PY{p}{,} \PY{n}{ascending}\PY{o}{=}\PY{k+kc}{False}\PY{p}{)}\PY{p}{[}\PY{l+m+mi}{0}\PY{p}{:}\PY{l+m+mi}{50}\PY{p}{]}
\PY{n}{topUsers}\PY{o}{.}\PY{n}{head}\PY{p}{(}\PY{p}{)}
\end{Verbatim}
\end{tcolorbox}

            \begin{tcolorbox}[breakable, size=fbox, boxrule=.5pt, pad at break*=1mm, opacityfill=0]
\prompt{Out}{outcolor}{23}{\boxspacing}
\begin{Verbatim}[commandchars=\\\{\}]
    similarityIndex  userId
64         0.961678   12325
34         0.961538    6207
55         0.961538   10707
67         0.960769   13053
4          0.943456    1040
\end{Verbatim}
\end{tcolorbox}
        
    Now, let's start recommending movies to the input user.

\hypertarget{rating-of-selected-users-to-all-movies}{%
\paragraph{Rating of selected users to all
movies}\label{rating-of-selected-users-to-all-movies}}

We're going to do this by taking the weighted average of the ratings of
the movies using the Pearson Correlation as the weight. But to do this,
we first need to get the movies watched by the users in our
\textbf{pearsonDF} from the ratings dataframe and then store their
correlation in a new column called \_similarityIndex". This is achieved
below by merging of these two tables.

    \begin{tcolorbox}[breakable, size=fbox, boxrule=1pt, pad at break*=1mm,colback=cellbackground, colframe=cellborder]
\prompt{In}{incolor}{24}{\boxspacing}
\begin{Verbatim}[commandchars=\\\{\}]
\PY{n}{topUsersRating}\PY{o}{=}\PY{n}{topUsers}\PY{o}{.}\PY{n}{merge}\PY{p}{(}\PY{n}{ratings\PYZus{}df}\PY{p}{,} \PY{n}{left\PYZus{}on}\PY{o}{=}\PY{l+s+s1}{\PYZsq{}}\PY{l+s+s1}{userId}\PY{l+s+s1}{\PYZsq{}}\PY{p}{,} \PY{n}{right\PYZus{}on}\PY{o}{=}\PY{l+s+s1}{\PYZsq{}}\PY{l+s+s1}{userId}\PY{l+s+s1}{\PYZsq{}}\PY{p}{,} \PY{n}{how}\PY{o}{=}\PY{l+s+s1}{\PYZsq{}}\PY{l+s+s1}{inner}\PY{l+s+s1}{\PYZsq{}}\PY{p}{)}
\PY{n}{topUsersRating}\PY{o}{.}\PY{n}{head}\PY{p}{(}\PY{p}{)}
\end{Verbatim}
\end{tcolorbox}

            \begin{tcolorbox}[breakable, size=fbox, boxrule=.5pt, pad at break*=1mm, opacityfill=0]
\prompt{Out}{outcolor}{24}{\boxspacing}
\begin{Verbatim}[commandchars=\\\{\}]
   similarityIndex  userId  movieId  rating
0         0.961678   12325        1     3.5
1         0.961678   12325        2     1.5
2         0.961678   12325        3     3.0
3         0.961678   12325        5     0.5
4         0.961678   12325        6     2.5
\end{Verbatim}
\end{tcolorbox}
        
    Now all we need to do is simply multiply the movie rating by its weight
(The similarity index), then sum up the new ratings and divide it by the
sum of the weights.

We can easily do this by simply multiplying two columns, then grouping
up the dataframe by movieId and then dividing two columns:

It shows the idea of all similar users to candidate movies for the input
user:

    \begin{tcolorbox}[breakable, size=fbox, boxrule=1pt, pad at break*=1mm,colback=cellbackground, colframe=cellborder]
\prompt{In}{incolor}{25}{\boxspacing}
\begin{Verbatim}[commandchars=\\\{\}]
\PY{c+c1}{\PYZsh{}Multiplies the similarity by the user\PYZsq{}s ratings}
\PY{n}{topUsersRating}\PY{p}{[}\PY{l+s+s1}{\PYZsq{}}\PY{l+s+s1}{weightedRating}\PY{l+s+s1}{\PYZsq{}}\PY{p}{]} \PY{o}{=} \PY{n}{topUsersRating}\PY{p}{[}\PY{l+s+s1}{\PYZsq{}}\PY{l+s+s1}{similarityIndex}\PY{l+s+s1}{\PYZsq{}}\PY{p}{]}\PY{o}{*}\PY{n}{topUsersRating}\PY{p}{[}\PY{l+s+s1}{\PYZsq{}}\PY{l+s+s1}{rating}\PY{l+s+s1}{\PYZsq{}}\PY{p}{]}
\PY{n}{topUsersRating}\PY{o}{.}\PY{n}{head}\PY{p}{(}\PY{p}{)}
\end{Verbatim}
\end{tcolorbox}

            \begin{tcolorbox}[breakable, size=fbox, boxrule=.5pt, pad at break*=1mm, opacityfill=0]
\prompt{Out}{outcolor}{25}{\boxspacing}
\begin{Verbatim}[commandchars=\\\{\}]
   similarityIndex  userId  movieId  rating  weightedRating
0         0.961678   12325        1     3.5        3.365874
1         0.961678   12325        2     1.5        1.442517
2         0.961678   12325        3     3.0        2.885035
3         0.961678   12325        5     0.5        0.480839
4         0.961678   12325        6     2.5        2.404196
\end{Verbatim}
\end{tcolorbox}
        
    \begin{tcolorbox}[breakable, size=fbox, boxrule=1pt, pad at break*=1mm,colback=cellbackground, colframe=cellborder]
\prompt{In}{incolor}{26}{\boxspacing}
\begin{Verbatim}[commandchars=\\\{\}]
\PY{c+c1}{\PYZsh{}Applies a sum to the topUsers after grouping it up by userId}
\PY{n}{tempTopUsersRating} \PY{o}{=} \PY{n}{topUsersRating}\PY{o}{.}\PY{n}{groupby}\PY{p}{(}\PY{l+s+s1}{\PYZsq{}}\PY{l+s+s1}{movieId}\PY{l+s+s1}{\PYZsq{}}\PY{p}{)}\PY{o}{.}\PY{n}{sum}\PY{p}{(}\PY{p}{)}\PY{p}{[}\PY{p}{[}\PY{l+s+s1}{\PYZsq{}}\PY{l+s+s1}{similarityIndex}\PY{l+s+s1}{\PYZsq{}}\PY{p}{,}\PY{l+s+s1}{\PYZsq{}}\PY{l+s+s1}{weightedRating}\PY{l+s+s1}{\PYZsq{}}\PY{p}{]}\PY{p}{]}
\PY{n}{tempTopUsersRating}\PY{o}{.}\PY{n}{columns} \PY{o}{=} \PY{p}{[}\PY{l+s+s1}{\PYZsq{}}\PY{l+s+s1}{sum\PYZus{}similarityIndex}\PY{l+s+s1}{\PYZsq{}}\PY{p}{,}\PY{l+s+s1}{\PYZsq{}}\PY{l+s+s1}{sum\PYZus{}weightedRating}\PY{l+s+s1}{\PYZsq{}}\PY{p}{]}
\PY{n}{tempTopUsersRating}\PY{o}{.}\PY{n}{head}\PY{p}{(}\PY{p}{)}
\end{Verbatim}
\end{tcolorbox}

            \begin{tcolorbox}[breakable, size=fbox, boxrule=.5pt, pad at break*=1mm, opacityfill=0]
\prompt{Out}{outcolor}{26}{\boxspacing}
\begin{Verbatim}[commandchars=\\\{\}]
         sum\_similarityIndex  sum\_weightedRating
movieId
1                  38.376281          140.800834
2                  38.376281           96.656745
3                  10.253981           27.254477
4                   0.929294            2.787882
5                  11.723262           27.151751
\end{Verbatim}
\end{tcolorbox}
        
    \begin{tcolorbox}[breakable, size=fbox, boxrule=1pt, pad at break*=1mm,colback=cellbackground, colframe=cellborder]
\prompt{In}{incolor}{27}{\boxspacing}
\begin{Verbatim}[commandchars=\\\{\}]
\PY{c+c1}{\PYZsh{}Creates an empty dataframe}
\PY{n}{recommendation\PYZus{}df} \PY{o}{=} \PY{n}{pd}\PY{o}{.}\PY{n}{DataFrame}\PY{p}{(}\PY{p}{)}
\PY{c+c1}{\PYZsh{}Now we take the weighted average}
\PY{n}{recommendation\PYZus{}df}\PY{p}{[}\PY{l+s+s1}{\PYZsq{}}\PY{l+s+s1}{weighted average recommendation score}\PY{l+s+s1}{\PYZsq{}}\PY{p}{]} \PY{o}{=} \PY{n}{tempTopUsersRating}\PY{p}{[}\PY{l+s+s1}{\PYZsq{}}\PY{l+s+s1}{sum\PYZus{}weightedRating}\PY{l+s+s1}{\PYZsq{}}\PY{p}{]}\PY{o}{/}\PY{n}{tempTopUsersRating}\PY{p}{[}\PY{l+s+s1}{\PYZsq{}}\PY{l+s+s1}{sum\PYZus{}similarityIndex}\PY{l+s+s1}{\PYZsq{}}\PY{p}{]}
\PY{n}{recommendation\PYZus{}df}\PY{p}{[}\PY{l+s+s1}{\PYZsq{}}\PY{l+s+s1}{movieId}\PY{l+s+s1}{\PYZsq{}}\PY{p}{]} \PY{o}{=} \PY{n}{tempTopUsersRating}\PY{o}{.}\PY{n}{index}
\PY{n}{recommendation\PYZus{}df}\PY{o}{.}\PY{n}{head}\PY{p}{(}\PY{p}{)}
\end{Verbatim}
\end{tcolorbox}

            \begin{tcolorbox}[breakable, size=fbox, boxrule=.5pt, pad at break*=1mm, opacityfill=0]
\prompt{Out}{outcolor}{27}{\boxspacing}
\begin{Verbatim}[commandchars=\\\{\}]
         weighted average recommendation score  movieId
movieId
1                                     3.668955        1
2                                     2.518658        2
3                                     2.657941        3
4                                     3.000000        4
5                                     2.316058        5
\end{Verbatim}
\end{tcolorbox}
        
    Now let's sort it and see the top 20 movies that the algorithm
recommended!

    \begin{tcolorbox}[breakable, size=fbox, boxrule=1pt, pad at break*=1mm,colback=cellbackground, colframe=cellborder]
\prompt{In}{incolor}{28}{\boxspacing}
\begin{Verbatim}[commandchars=\\\{\}]
\PY{n}{recommendation\PYZus{}df} \PY{o}{=} \PY{n}{recommendation\PYZus{}df}\PY{o}{.}\PY{n}{sort\PYZus{}values}\PY{p}{(}\PY{n}{by}\PY{o}{=}\PY{l+s+s1}{\PYZsq{}}\PY{l+s+s1}{weighted average recommendation score}\PY{l+s+s1}{\PYZsq{}}\PY{p}{,} \PY{n}{ascending}\PY{o}{=}\PY{k+kc}{False}\PY{p}{)}
\PY{n}{recommendation\PYZus{}df}\PY{o}{.}\PY{n}{head}\PY{p}{(}\PY{l+m+mi}{10}\PY{p}{)}
\end{Verbatim}
\end{tcolorbox}

            \begin{tcolorbox}[breakable, size=fbox, boxrule=.5pt, pad at break*=1mm, opacityfill=0]
\prompt{Out}{outcolor}{28}{\boxspacing}
\begin{Verbatim}[commandchars=\\\{\}]
         weighted average recommendation score  movieId
movieId
5073                                       5.0     5073
3329                                       5.0     3329
2284                                       5.0     2284
26801                                      5.0    26801
6776                                       5.0     6776
6672                                       5.0     6672
3759                                       5.0     3759
3769                                       5.0     3769
3775                                       5.0     3775
90531                                      5.0    90531
\end{Verbatim}
\end{tcolorbox}
        
    \begin{tcolorbox}[breakable, size=fbox, boxrule=1pt, pad at break*=1mm,colback=cellbackground, colframe=cellborder]
\prompt{In}{incolor}{29}{\boxspacing}
\begin{Verbatim}[commandchars=\\\{\}]
\PY{n}{movies\PYZus{}df}\PY{o}{.}\PY{n}{loc}\PY{p}{[}\PY{n}{movies\PYZus{}df}\PY{p}{[}\PY{l+s+s1}{\PYZsq{}}\PY{l+s+s1}{movieId}\PY{l+s+s1}{\PYZsq{}}\PY{p}{]}\PY{o}{.}\PY{n}{isin}\PY{p}{(}\PY{n}{recommendation\PYZus{}df}\PY{o}{.}\PY{n}{head}\PY{p}{(}\PY{l+m+mi}{10}\PY{p}{)}\PY{p}{[}\PY{l+s+s1}{\PYZsq{}}\PY{l+s+s1}{movieId}\PY{l+s+s1}{\PYZsq{}}\PY{p}{]}\PY{o}{.}\PY{n}{tolist}\PY{p}{(}\PY{p}{)}\PY{p}{)}\PY{p}{]}
\end{Verbatim}
\end{tcolorbox}

            \begin{tcolorbox}[breakable, size=fbox, boxrule=.5pt, pad at break*=1mm, opacityfill=0]
\prompt{Out}{outcolor}{29}{\boxspacing}
\begin{Verbatim}[commandchars=\\\{\}]
       movieId                                    title  year
2200      2284                             Bandit Queen  1994
3243      3329                 Year My Voice Broke, The  1987
3669      3759                       Fun and Fancy Free  1947
3679      3769                Thunderbolt and Lightfoot  1974
3685      3775                          Make Mine Music  1946
4978      5073  Son's Room, The (Stanza del figlio, La)  2001
6563      6672                         War Photographer  2001
6667      6776        Lagaan: Once Upon a Time in India  2001
9064     26801      Dragon Inn (Sun lung moon hak chan)  1992
18106    90531                                    Shame  2011
\end{Verbatim}
\end{tcolorbox}
        
    \hypertarget{advantages-and-disadvantages-of-collaborative-filtering}{%
\subsubsection{Advantages and Disadvantages of Collaborative
Filtering}\label{advantages-and-disadvantages-of-collaborative-filtering}}

\hypertarget{advantages}{%
\subparagraph{Advantages}\label{advantages}}

\begin{itemize}
\tightlist
\item
  Takes other user's ratings into consideration
\item
  Doesn't need to study or extract information from the recommended item
\item
  Adapts to the user's interests which might change over time
\end{itemize}

\hypertarget{disadvantages}{%
\subparagraph{Disadvantages}\label{disadvantages}}

\begin{itemize}
\tightlist
\item
  Approximation function can be slow
\item
  There might be a low of amount of users to approximate
\item
  Privacy issues when trying to learn the user's preferences
\end{itemize}

    Want to learn more?

IBM SPSS Modeler is a comprehensive analytics platform that has many
machine learning algorithms. It has been designed to bring predictive
intelligence to decisions made by individuals, by groups, by systems --
by your enterprise as a whole. A free trial is available through this
course, available here: SPSS Modeler

Also, you can use Watson Studio to run these notebooks faster with
bigger datasets. Watson Studio is IBM's leading cloud solution for data
scientists, built by data scientists. With Jupyter notebooks, RStudio,
Apache Spark and popular libraries pre-packaged in the cloud, Watson
Studio enables data scientists to collaborate on their projects without
having to install anything. Join the fast-growing community of Watson
Studio users today with a free account at Watson Studio

Thanks for completing this lesson!

Author: Saeed Aghabozorgi

Saeed Aghabozorgi, PhD is a Data Scientist in IBM with a track record of
developing enterprise level applications that substantially increases
clients' ability to turn data into actionable knowledge. He is a
researcher in data mining field and expert in developing advanced
analytic methods like machine learning and statistical modelling on
large datasets.

Copyright © 2018 Cognitive Class. This notebook and its source code are
released under the terms of the MIT License.


    % Add a bibliography block to the postdoc
    
    
    
\end{document}
